\documentclass[12pt,a4paper]{article}

\usepackage[utf8]{inputenc} % Codificación de caracteres
\usepackage{amsmath} % Símbolos matemáticos
\usepackage{amssymb}
\usepackage[a4paper, top=1cm, bottom=3cm, left=2.5cm, right=2.5cm]{geometry}


\title{Final Algebra 04/03/2024}
\author{Nahuel Prieto \\ \texttt{enprieto@dc.uba.ar}} 
\date{}
\begin{document}

\maketitle

\section{Ejercicio 1}
    Se define por recurrencia la sucesión $(a_{n})_{n \in \mathbb{N}}$ de la siguente manera 
\begin{align*}
    a_1 &= 4,  a_2 = 5 \\ a_{n+2} &= a_{n+1} - 10a_n \hspace{0.5cm} \forall n \in \mathbb{N}
\end{align*}
Probar que $a_n\equiv 4^n \pmod{11}$ $\forall n \in \mathbb{N}$ y calcular el resto \\
de la division por 11 de $\sum_{n=1}^{2024} a_n$\\ \\
\textbf{Solucion:} \\
Probamos por induccion
\begin{align*}   
    P_{(n)}: a_n \equiv 4^n \pmod{11}
\end{align*}
Caso base: ¿$P_{(1)}$ verdadera? Sí, pues $4 \equiv 4 \pmod{11}$\\
\hspace*{50px}¿$P_{(2)}$ verdadera? Sí, pues $5 \equiv 4^2 \pmod{11}$ \\ \\
Paso Inductivo: $\forall h \in \mathbb{N}$, ¿$P_{(h)}$ $\land$ $P_{(h+1)}$ verdadera? $\Rightarrow$ ¿$P_{(h+2)}$ verdadera? \\ \\
HI: $a_h \equiv 4^h \pmod{11}$ $\land$ $a_{h+1} \equiv 4^{h+1} \pmod{11}$ \\
Qpq: $a_{h+2} \equiv 4^{h+2} \pmod{11}$ \\ \\
obs: $a_h \equiv 4^h (11) \quad \Longleftrightarrow \quad a_h = 11k +4^h, \quad k \in \mathbb{Z}$  \\
\hspace*{0.4cm}$a_{h+1} \equiv 4^{h+1} (11) \quad \Longleftrightarrow \quad a_{h+1} = 11j +4^{h+1}, \quad j \in \mathbb{Z}$ 
por def: $a_{h+2} = a_{h+1} - 10a_h$     
\begin{align*}  
\therefore \quad a_{h+2} &\underset{\scriptscriptstyle\text{HI}}{=} (11j + 4^{h+1}) - 10(11k + 4^h) \\
    a_{h+2} &= 11j + 4^{h+1} - 10.11k - 10(4^h) \\
    a_{h+2} &= 11 \underbrace{(j - 10k)}_{Q} - 6 \cdot 4^h \\
    a_{h+2} +6 \cdot 4^h &= 11Q \quad \Longleftrightarrow \quad 11 \mid a_{h+2} +6 \cdot 4^h \\ \\
    \therefore \quad a_{h+2} +6 \cdot 4^h &\equiv 0 (11)\\
    a_{h+2} &\equiv -6 \cdot 4^h (11) \\
    -a_{h+2} &\underset{\scriptscriptstyle(-1)}{\equiv}  6.4^h (11) \\
    -2a_{h+2} &\underset{\scriptscriptstyle(2)}{\equiv} 4^h (11) \\
    9a_{h+2} &\equiv 4^h (11) \\
    144a_{h+2} &\underset{\scriptscriptstyle(4^2)}{\equiv}  4^h \cdot 4^2 (11) \quad
    \Longleftrightarrow \quad  a_{h+2} \equiv 4^{h+2} (11) \quad 
    \text{tal como se queria probar} 
\end{align*}
obs: $144 = 11 \cdot 13 + 1 \quad \Longleftrightarrow   \quad 144 \equiv 1 (11)$ \\
Como se probo el caso base y el paso inductivo, se concluye que \\
$P_{(n)}$ es verdadera $\forall n \in \mathbb{N}$ \\ \\
Ahora voy a calcular el resto de dividir \\
\begin{equation*}
\sum_{n=1}^{2024} a_n \quad \text{por} \quad 11
\end{equation*}
Probamos anteriormente  que $a_n \equiv 4^n (11)$ \\
\begin{equation*}
\therefore \quad \sum_{n=1}^{2024}a_n  \quad = \quad  \sum_{n=1}^{2024} 4^n
\end{equation*}
Me interesa saber $ \quad r_{\scriptscriptstyle 11} (4^n)  \quad \forall n \in \mathbb{N}$ \\
voy aplicar PTF, $11 \nmid 4 \quad \Longrightarrow \quad 4^n \equiv 4^{r_{\scriptscriptstyle 10} (n)} (11)$ \\ \\
\begin{tabular}{|c|c|c|c|c|c|c|c|c|c|c|}
    \hline
    $r_{10}(n) $ & 0 & 1 & 2 & 3 & 4 & 5 & 6 & 7 & 8 & 9 \\
    \hline
    $r_{11}(4^{r_{10}(n)})$ & 1 & 4 & 5 & 9 & 3 & 1 & 4 & 5 & 9 & 3 \\
    \hline
\end{tabular} \\ \\ \\
Viendo la tabla de restos se puede observar que cada 5 sucesiones se vuelven \\
a repetir los mimos restos
\begin{align*}
&\therefore \quad \sum_{n=1}^{2024} \underbrace{4^1 + 4^2 + 4^3 + 4^4 + 4^5}_{\text{5 sucesiones}}  + \ldots \ldots  
+ 4^{2024} \\ 
&4^1 + 4^2 + 4^3 + 4^4 + 4^5 \equiv 4 + 5 + 9 + 3 + 1 \equiv 22 \equiv 0 (11) 
\end{align*}
Voy a pensar las 5 sucesiones como que son un bloque, en $2024 = 5.404 + 4$ \\
es decir tengo 404 bloques de 5 sucesiones todos esos bloques son congruentes a $0 (11)$.
Por lo que en la sumatoria sobreviven $4^{2021} + 4^{2022} + 4^{2023} + 4^{2024}$ \\
que por tabla de restos se que: \\ \\ $4^{2021} + 4^{2022} + 4^{2023} + 4^{2024} \equiv 4 + 5
+ 9 + 3 \equiv 21 \equiv 10 \quad (11)$ \\ \\
$\therefore \quad \sum_{n=1}^{2024} a_n \equiv \sum_{n=1}^{2024} 4^n \equiv 10 \quad (11)$  \hspace{15px}se concluye que  $r_{11}(\sum_{n=1}^{2024} a_n) = 10$

\section{Ejercicio 2}
a) Enunciar la definicion de divisibilidad para enteros y determinar si la relacion \\
$\Re$ sobre $\mathbb{Z}$ dada por
\begin{align*}
    a \Re b \quad \Longleftrightarrow \quad a \mid b \quad \forall a,b \in \mathbb{Z}, \quad a \neq 0
\end{align*}
es reflexiva, simetrica, antisimetrica y transitiva. \\ \\
b) Calcular el cardinal del conjunto
\begin{align*}
    \{a \in \mathbb{Z}: a \hspace{3px} \Re \hspace{2px}14580000 \quad \land \quad a \equiv 0 \quad(15)\}
\end{align*}
\textbf{Solucion: a)} \\

\end{document}